\documentclass[twoside, a4paper]{article}
\usepackage{fontenc}
\usepackage{inputenc}
\usepackage[english,russian]{babel}
\usepackage{amsmath}
\usepackage{amsfonts}
\usepackage{amssymb}
\usepackage{makeidx}
\usepackage{lipsum}
\usepackage{fancyhdr} 
\usepackage[
    top=30mm,
    bottom=20mm,
    left=20mm,
    right=20mm
]{geometry}
\pagestyle{fancy} 
\fancyhf{}
\fancyhead[CE]{4.8 Гамильтоновы система с двумя степенями свободы \ldots} 
\fancyhead[CO]{Глава 4}
\fancyhead[R]{\thepage}

\begin{document}
\newcounter{taskc}
\newcounter{theoremc}
\newcommand{\task}[1]{\par\refstepcounter{taskc} \texttt{Упражнение 4.8.\thetaskc} {\small#1}}
\task{
Рассмотрите <<нелинейный гармонический>> осциллятор с гамильтонианом\\ $H=\frac13(p_1^2+q_1^2)^{3/2}+\frac12(p_2^2+q_2^2)$. Покажите, что редуцированную систему можно записать в виде 
$$q_1'=p_1\sqrt{p_1^2+q_1^2},$$
$$p_1=-q_1]\sqrt{p_1^2+q_1^2},$$
откуда сделайте вывод, что отображание Пуанкаре имеет два плотных множества замкнутых кривых, одно из которых заполнено периодическими орбитами, другое --- плотными орбитами.
}\label{red_sys}
\task{
Проведите редукцию модели <<маятник-осциллятор>> с гамильтонианом
$$H(q_1,p_1,q_2,p_2)=\frac{p_1^2}{2}+(1-\cos q_1)+\frac{p_2^2+\omega_2^2g_2^2}{2}$$
и обсудите соответствующее отображение Пуанкаре. (Мы вернёмся к этому примеру позже в данном разделе.)
}\label{osc}

Примеры, которые мы обсуждали до сих пор, являются \textit{вполне интегрируемыми} в том смысле, что имеется \textit{две независимые} функции (одна из которых $H$), остающиеся инвариантными относительно потока, определяемого уранвениями Гамильтона. Для первого из примеров в качестве второй функции можно взять действие для второго из осцилляторов
$$I=(p_2^2+\omega_2^2q_2^2)/2\omega_2,$$
при этом решения лежат на двумерных торах, представляющих собой пересечения поверхностей
\renewcommand{\theequation}{4.8.\arabic{equation}}
\setcounter{equation}{17}
\begin{equation}
H = h^0,\text{~~} I=I^0
\end{equation}

Аналогично можно рассмотреть примеры из упражнений 4.8.\ref{red_sys}--4.8.\ref{osc} Однако, как понимали Пуанкаре и Биркгоф, очень немногие системы с двумя степенями свободы обладают двумя независимыми интегралами, и классическая теория Гамильтона-Якоби, связанная с отысканием такт интегралов, в болыиинстве случаев не срабатывает. Более того, попытки приближенного вычисления второго интеграла методами усреднения или теории возмущений, а также путем вычисления нормальной формы в такт случаях не приводят к успеху. Мы отсылаем читателя к книге Lichtenberg, Lieberman [1982], где обсуждаются эти методы, огранияиваясь здесь демонстрацией того, как метод Мельникова в сочетании с редукцией может быть использован для доказательства несуществования второго (аналитического) интеграла движения в конкретных примерах. При обсуждении этого метода мы сделаем также ряд общих наблюдений о свойствах двумерного отображения Пуанкаре $P_{h^0}^{\Theta_0}$ редуцированной системы. В дальнейшем мы будем опускать индексы $h^0$, $\Theta_0$ и  писать $P_{h^0}^{\Theta_0}=P_\epsilon$ или $P_0$, где индексы свидетельствуют о наличии или отсутствии возмущения $\epsilon H^1$ в нижеследующем уравнении (\ref{eq:label1}).
Вначале заметим, что $P$ --- сохраняющий площадь диффеоморфизм, так
как
%$$-\frac{\partial^2L}{\partial q\partial p}$$
\begin{equation}
DP=e^{2\pi Df} , \text{~~}  Df=
\left[\begin{array}{cc}
-\frac{\partial^2L}{\partial q\partial p} & -\frac{\partial^2L}{\partial p^2} \\
-\frac{\partial^2L}{\partial q\partial p} & -\frac{\partial^2L}{\partial p^2}
\end{array}\right]
\end{equation}
и след матрицы $Df$ тождественно равен нулю. (Сохранение объёма, в силу теоремы Лиувилля, проявляется здесь в сохранении площади при отображении $P$.)
Предположим далее, что наш гамильтониан является (малым) возму- щением $H^\epsilon$ некоторого интегрируемого гамильтониана $H^0$ . Возьмем для простоты систему вида
\begin{equation}
H^\epsilon (q,p,\Theta,I) = F(q,p)+G(I)+\epsilon H^1(q,p,\Theta, I),
\label{eq:label1}
\end{equation}

где функция $ H^1 $ имеет по $\Theta$ период $ 2\pi $, а невозмущённая система $H^0(q,p,\Theta,I)=F(q,p)+G(I)$ непосредственно распадается на две независимые системы с интегралами $F$ и $G$ (или, что равносильно, $H^0$ и $I$. Как и ранее, сделаем предпололжение о невырожденности
\newcommand*{\defeq}{\stackrel{\text{def}}{=}}
\begin{equation}
\Omega(I) \defeq \frac{\partial G}{\partial I} \left(=\frac{\partial H^0}{\partial I}\right) \neq 0
\end{equation}
конкретнее, $\Omega>0$ при $I>0$. Отсюда следует, что для малых $\epsilon$ уравнение $H^\epsilon = F+G+\epsilon H^1=h$ разрешимо относительно I как и ранее. Однако наличие здесь малого параметра $\epsilon$ позволяет нам вычислить обратую функцию $L$ в уравнении (4.8.7) %оно не в моей главе
явно в виде рядов по степеням $\epsilon$. Мы имеем 
\begin{equation}
\text{~}\begin{aligned}
&I=L^\epsilon (q,p,\Theta;h) = L^0(q,p;h)+\epsilon L^1(q,p,\Theta;h) + O(\epsilon^2),\\
&l^0=G^{-1}(h-F(q,p)),\\
&L^1=-\frac{H^1(q,p,\Theta,L^0(q,p;h))}{\Omega(L^0(q,p;h))}
\end{aligned}
\label{eq:label2}
\end{equation}
\task{проверьте соотношения (\ref{eq:label2})}

В силу (\ref{eq:label2}) редуцированная гамильтонова система принимает вид
\begin{equation}
\text{~}\begin{aligned}
&q'=-\frac{\partial L^0}{\partial p}(q,p;h)-\epsilon\frac{\partial L^1}{\partial p}(q,p,\Theta;h)+O(\epsilon^2),\\
&p'=\frac{\partial L^0}{\partial q}(q,p;h)+\epsilon\frac{\partial L^1}{\partial q}(q,p,\Theta;h)+O(\epsilon^2),\\
\end{aligned}
\label{eq:label3}
\end{equation}
Поскольку $H^1$ является $2\pi$-периодической функцией $\Theta$, такова же и $L^1$, так что система (\ref{eq:label3}) относится именно к тому типу, который изучался в предыдущем разделе при помощи метода Мельникова.

В частности, допустим (как выше), что некоторая (компактная) область фазовой плоскости для системы $F(q,p)$ заполнена периодическими орбитами, периоды которых монотонно изменяются при изменении энергии $F$. Каждая такая орбита является множеством уровня для $F:F(q,p)=h^\alpha$, следовательно, если общая энергия $h$ больше,чем $h^\alpha$, невозмущённая $(\epsilon=0)$
%\lipsum[1-80]
\end{document}

